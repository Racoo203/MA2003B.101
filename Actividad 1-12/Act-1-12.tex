% Options for packages loaded elsewhere
\PassOptionsToPackage{unicode}{hyperref}
\PassOptionsToPackage{hyphens}{url}
%
\documentclass[
]{article}
\usepackage{amsmath,amssymb}
\usepackage{iftex}
\ifPDFTeX
  \usepackage[T1]{fontenc}
  \usepackage[utf8]{inputenc}
  \usepackage{textcomp} % provide euro and other symbols
\else % if luatex or xetex
  \usepackage{unicode-math} % this also loads fontspec
  \defaultfontfeatures{Scale=MatchLowercase}
  \defaultfontfeatures[\rmfamily]{Ligatures=TeX,Scale=1}
\fi
\usepackage{lmodern}
\ifPDFTeX\else
  % xetex/luatex font selection
\fi
% Use upquote if available, for straight quotes in verbatim environments
\IfFileExists{upquote.sty}{\usepackage{upquote}}{}
\IfFileExists{microtype.sty}{% use microtype if available
  \usepackage[]{microtype}
  \UseMicrotypeSet[protrusion]{basicmath} % disable protrusion for tt fonts
}{}
\makeatletter
\@ifundefined{KOMAClassName}{% if non-KOMA class
  \IfFileExists{parskip.sty}{%
    \usepackage{parskip}
  }{% else
    \setlength{\parindent}{0pt}
    \setlength{\parskip}{6pt plus 2pt minus 1pt}}
}{% if KOMA class
  \KOMAoptions{parskip=half}}
\makeatother
\usepackage{xcolor}
\usepackage[margin=1in]{geometry}
\usepackage{color}
\usepackage{fancyvrb}
\newcommand{\VerbBar}{|}
\newcommand{\VERB}{\Verb[commandchars=\\\{\}]}
\DefineVerbatimEnvironment{Highlighting}{Verbatim}{commandchars=\\\{\}}
% Add ',fontsize=\small' for more characters per line
\usepackage{framed}
\definecolor{shadecolor}{RGB}{248,248,248}
\newenvironment{Shaded}{\begin{snugshade}}{\end{snugshade}}
\newcommand{\AlertTok}[1]{\textcolor[rgb]{0.94,0.16,0.16}{#1}}
\newcommand{\AnnotationTok}[1]{\textcolor[rgb]{0.56,0.35,0.01}{\textbf{\textit{#1}}}}
\newcommand{\AttributeTok}[1]{\textcolor[rgb]{0.13,0.29,0.53}{#1}}
\newcommand{\BaseNTok}[1]{\textcolor[rgb]{0.00,0.00,0.81}{#1}}
\newcommand{\BuiltInTok}[1]{#1}
\newcommand{\CharTok}[1]{\textcolor[rgb]{0.31,0.60,0.02}{#1}}
\newcommand{\CommentTok}[1]{\textcolor[rgb]{0.56,0.35,0.01}{\textit{#1}}}
\newcommand{\CommentVarTok}[1]{\textcolor[rgb]{0.56,0.35,0.01}{\textbf{\textit{#1}}}}
\newcommand{\ConstantTok}[1]{\textcolor[rgb]{0.56,0.35,0.01}{#1}}
\newcommand{\ControlFlowTok}[1]{\textcolor[rgb]{0.13,0.29,0.53}{\textbf{#1}}}
\newcommand{\DataTypeTok}[1]{\textcolor[rgb]{0.13,0.29,0.53}{#1}}
\newcommand{\DecValTok}[1]{\textcolor[rgb]{0.00,0.00,0.81}{#1}}
\newcommand{\DocumentationTok}[1]{\textcolor[rgb]{0.56,0.35,0.01}{\textbf{\textit{#1}}}}
\newcommand{\ErrorTok}[1]{\textcolor[rgb]{0.64,0.00,0.00}{\textbf{#1}}}
\newcommand{\ExtensionTok}[1]{#1}
\newcommand{\FloatTok}[1]{\textcolor[rgb]{0.00,0.00,0.81}{#1}}
\newcommand{\FunctionTok}[1]{\textcolor[rgb]{0.13,0.29,0.53}{\textbf{#1}}}
\newcommand{\ImportTok}[1]{#1}
\newcommand{\InformationTok}[1]{\textcolor[rgb]{0.56,0.35,0.01}{\textbf{\textit{#1}}}}
\newcommand{\KeywordTok}[1]{\textcolor[rgb]{0.13,0.29,0.53}{\textbf{#1}}}
\newcommand{\NormalTok}[1]{#1}
\newcommand{\OperatorTok}[1]{\textcolor[rgb]{0.81,0.36,0.00}{\textbf{#1}}}
\newcommand{\OtherTok}[1]{\textcolor[rgb]{0.56,0.35,0.01}{#1}}
\newcommand{\PreprocessorTok}[1]{\textcolor[rgb]{0.56,0.35,0.01}{\textit{#1}}}
\newcommand{\RegionMarkerTok}[1]{#1}
\newcommand{\SpecialCharTok}[1]{\textcolor[rgb]{0.81,0.36,0.00}{\textbf{#1}}}
\newcommand{\SpecialStringTok}[1]{\textcolor[rgb]{0.31,0.60,0.02}{#1}}
\newcommand{\StringTok}[1]{\textcolor[rgb]{0.31,0.60,0.02}{#1}}
\newcommand{\VariableTok}[1]{\textcolor[rgb]{0.00,0.00,0.00}{#1}}
\newcommand{\VerbatimStringTok}[1]{\textcolor[rgb]{0.31,0.60,0.02}{#1}}
\newcommand{\WarningTok}[1]{\textcolor[rgb]{0.56,0.35,0.01}{\textbf{\textit{#1}}}}
\usepackage{graphicx}
\makeatletter
\def\maxwidth{\ifdim\Gin@nat@width>\linewidth\linewidth\else\Gin@nat@width\fi}
\def\maxheight{\ifdim\Gin@nat@height>\textheight\textheight\else\Gin@nat@height\fi}
\makeatother
% Scale images if necessary, so that they will not overflow the page
% margins by default, and it is still possible to overwrite the defaults
% using explicit options in \includegraphics[width, height, ...]{}
\setkeys{Gin}{width=\maxwidth,height=\maxheight,keepaspectratio}
% Set default figure placement to htbp
\makeatletter
\def\fps@figure{htbp}
\makeatother
\setlength{\emergencystretch}{3em} % prevent overfull lines
\providecommand{\tightlist}{%
  \setlength{\itemsep}{0pt}\setlength{\parskip}{0pt}}
\setcounter{secnumdepth}{-\maxdimen} % remove section numbering
\ifLuaTeX
  \usepackage{selnolig}  % disable illegal ligatures
\fi
\usepackage{bookmark}
\IfFileExists{xurl.sty}{\usepackage{xurl}}{} % add URL line breaks if available
\urlstyle{same}
\hypersetup{
  pdftitle={Actividad 1.12 Series de tiempo no estacionarias},
  pdfauthor={Raúl Correa Ocañas},
  hidelinks,
  pdfcreator={LaTeX via pandoc}}

\title{Actividad 1.12 Series de tiempo no estacionarias}
\author{Raúl Correa Ocañas}
\date{2024-08-31}

\begin{document}
\maketitle

\section{Resumen Breve del Método
Predictivo}\label{resumen-breve-del-muxe9todo-predictivo}

El documento de Markdown se centra en el análisis de los niveles de
monóxido de carbono (CO) registrados en la estación TLA desde el año
2000 hasta 2022. Primero se cargan los CSVs anuales de los años 2000,
2010, 2021 y 2022. En la limpieza de los datos, se sustituyen los
valores faltantes (\texttt{-99)} con \texttt{NA}, para facilitar el
manejo de la información. Además, se realiza una segmentación de los
datos por mes a nivel día y hora. Una vez hecha esta agrupación, se
realiza el cálculo de los promedios mensuales de las concentraciones de
CO. Utilizando estas medias, se generan gráficos que visualizan las
tendencias a lo largo del tiempo, permitiendo observar variaciones
estacionales.

El análisis también incluye la creación de una serie de tiempo a partir
de los promedios mensuales de CO. Esta serie temporal se descompone en
sus componentes de tendencia, estacionalidad y residuos, utilizando la
función \texttt{decompose}. Se trabaja sobre una versión
desestacionalizada de la serie de tiempo. Sobre esta, primero se prueba
un modelo de regresión lineal para modelar la serie de tiempo. Bien si
sus coeficientes son significativos al llamar la función
\texttt{summary}, en verdad no representa del todo bien los datos. El
modelo cuadrático representa mejor tanto con su métrica de \(R^2\) como
con la significancia de sus coeficientes.

\begin{Shaded}
\begin{Highlighting}[]
\CommentTok{\# Cargar datos y crear la serie temporal}
\NormalTok{CO }\OtherTok{\textless{}{-}} \FunctionTok{ts}\NormalTok{(}\FunctionTok{read.csv}\NormalTok{(}\StringTok{\textquotesingle{}../data/mediasCO2000\_2022.csv\textquotesingle{}}\NormalTok{)[}\DecValTok{1}\SpecialCharTok{:}\DecValTok{47}\NormalTok{,], }\AttributeTok{frequency =} \DecValTok{12}\NormalTok{, }\AttributeTok{start =} \FunctionTok{c}\NormalTok{(}\DecValTok{2000}\NormalTok{, }\DecValTok{1}\NormalTok{))}

\CommentTok{\# Descomposición de la serie}
\NormalTok{T }\OtherTok{\textless{}{-}} \FunctionTok{decompose}\NormalTok{(CO, }\AttributeTok{type =} \StringTok{"multiplicative"}\NormalTok{)}

\CommentTok{\# Modelos de regresión lineal y cuadrático}
\NormalTok{f1 }\OtherTok{\textless{}{-}} \ControlFlowTok{function}\NormalTok{(mes) }\FloatTok{2.22998} \SpecialCharTok{{-}} \FloatTok{0.04908} \SpecialCharTok{*}\NormalTok{ mes}
\NormalTok{f2 }\OtherTok{\textless{}{-}} \ControlFlowTok{function}\NormalTok{(mes) }\FloatTok{3.028454} \SpecialCharTok{{-}} \FloatTok{0.146851} \SpecialCharTok{*}\NormalTok{ mes }\SpecialCharTok{+} \FloatTok{0.002037} \SpecialCharTok{*}\NormalTok{ mes}\SpecialCharTok{\^{}}\DecValTok{2}

\CommentTok{\# Predicciones ajustadas y errores}
\NormalTok{g1 }\OtherTok{\textless{}{-}} \FunctionTok{f1}\NormalTok{(}\DecValTok{1}\SpecialCharTok{:}\DecValTok{47}\NormalTok{) }\SpecialCharTok{*}\NormalTok{ T}\SpecialCharTok{$}\NormalTok{seasonal[}\DecValTok{1}\SpecialCharTok{:}\DecValTok{47}\NormalTok{]}
\NormalTok{g2 }\OtherTok{\textless{}{-}} \FunctionTok{f2}\NormalTok{(}\DecValTok{1}\SpecialCharTok{:}\DecValTok{47}\NormalTok{) }\SpecialCharTok{*}\NormalTok{ T}\SpecialCharTok{$}\NormalTok{seasonal[}\DecValTok{1}\SpecialCharTok{:}\DecValTok{47}\NormalTok{]}
\NormalTok{e1 }\OtherTok{\textless{}{-}}\NormalTok{ CO }\SpecialCharTok{{-}}\NormalTok{ g1}
\NormalTok{e2 }\OtherTok{\textless{}{-}}\NormalTok{ CO }\SpecialCharTok{{-}}\NormalTok{ g2}

\CommentTok{\# Calcular CME y MAPE para ambos modelos y mostrar resultados}
\FunctionTok{cat}\NormalTok{(}
  \StringTok{"El CME del Método de proyección de tendencia Lineal es:"}\NormalTok{, }\FunctionTok{round}\NormalTok{(}\FunctionTok{mean}\NormalTok{(e1}\SpecialCharTok{\^{}}\DecValTok{2}\NormalTok{, }\AttributeTok{na.rm =} \ConstantTok{TRUE}\NormalTok{), }\DecValTok{3}\NormalTok{), }\StringTok{"}\SpecialCharTok{\textbackslash{}n}\StringTok{"}\NormalTok{,}
  \StringTok{"El MAPE del Método de proyección de tendencia Lineal es:"}\NormalTok{, }\FunctionTok{round}\NormalTok{(}\FunctionTok{mean}\NormalTok{(}\FunctionTok{abs}\NormalTok{(e1 }\SpecialCharTok{/}\NormalTok{ g1), }\AttributeTok{na.rm =} \ConstantTok{TRUE}\NormalTok{) }\SpecialCharTok{*} \DecValTok{100}\NormalTok{, }\DecValTok{3}\NormalTok{), }\StringTok{"\%}\SpecialCharTok{\textbackslash{}n}\StringTok{"}\NormalTok{,}
  \StringTok{"El CME del Método de proyección de tendencia Cuadrática es:"}\NormalTok{, }\FunctionTok{round}\NormalTok{(}\FunctionTok{mean}\NormalTok{(e2}\SpecialCharTok{\^{}}\DecValTok{2}\NormalTok{, }\AttributeTok{na.rm =} \ConstantTok{TRUE}\NormalTok{), }\DecValTok{3}\NormalTok{), }\StringTok{"}\SpecialCharTok{\textbackslash{}n}\StringTok{"}\NormalTok{,}
  \StringTok{"El MAPE del Método de proyección de tendencia Cuadrática es:"}\NormalTok{, }\FunctionTok{round}\NormalTok{(}\FunctionTok{mean}\NormalTok{(}\FunctionTok{abs}\NormalTok{(e2 }\SpecialCharTok{/}\NormalTok{ g2), }\AttributeTok{na.rm =} \ConstantTok{TRUE}\NormalTok{) }\SpecialCharTok{*} \DecValTok{100}\NormalTok{, }\DecValTok{3}\NormalTok{), }\StringTok{"\%}\SpecialCharTok{\textbackslash{}n}\StringTok{"}
\NormalTok{)}
\end{Highlighting}
\end{Shaded}

\begin{verbatim}
## El CME del Método de proyección de tendencia Lineal es: 0.12 
##  El MAPE del Método de proyección de tendencia Lineal es: 150.521 %
##  El CME del Método de proyección de tendencia Cuadrática es: 0.023 
##  El MAPE del Método de proyección de tendencia Cuadrática es: 12.539 %
\end{verbatim}

\end{document}
